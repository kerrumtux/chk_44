\section{Техническое задание}
\subsection{Основание для разработки}

Полное наименование системы: "Интерпретатор функционального языка программирования с поддержкой метапрограммирования".

Основанием для разработки программы является приказ ректора ЮЗГУ от «15» апреля 2024 г. №1779-с «Об утверждении тем выпускных квалификационных работ».

\subsection{Цель и назначение разработки}

Цель этой работы - разработка программной системы, позволяющей сокращение размера исходного кода программ за счёт метапрограммирования.

Для достижения этой цели было принято решение разработать интерпретатор функционального языка программирования с поддержкой метапрограммирования, который будет называться демонстрационным языком программирования (ДЯП) в рамках этой работы. Основной задачей этой работы является разработка программного обеспечения, способного анализировать и исполнять программы, написанные на функциональном языке программирования, а также обеспечивать возможности генерации и изменения кода на этом языке с использованием инструментов метапрограммирования.

Интерпретатор, созданный в рамках данной работы, должен иметь все ключевые функции, обеспечивающие поддержку парадигм метапрограммирования и функционального программирования. Для их реализации будет разработан простой и минималистичный функциональный язык программирования, называемый ДЯП, поддерживающий метапрограммирование. 

Таким образом, интерпретатор сможет работать с  числами, строками, переменными, функциями, лямбда-выражениями, макросами и другими необходимыми конструкциями, обеспечивающими разработчику возможность использовать метапрограммирование для создания адаптивных и реплицируемых приложений.

Задачами данной разработки являются:
\begin{itemize}
\item разработка синтаксиса ДЯП, достаточного для реализации функционального программирования и метапрограммирования;
\item разработка объектов, используемых для представления интерпретируемого исходного кода внутри интерпретатора;
\item разработка сборщика мусора;
\item разработка лексического анализатора для созданного языка;
\item разработка синтаксического анализатор для созданного языка;
\item разработка исполнителя инструкций;
\item реализация примитивных функций созданного языка.

\end{itemize}

\subsection{Требования к оформлению документации}

Разработка программной документации и программного изделия должна производиться согласно ГОСТ 19.102-77 и ГОСТ 34.601-90. Единая система программной документации.
