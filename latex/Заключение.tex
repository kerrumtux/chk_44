\section*{ЗАКЛЮЧЕНИЕ}
\addcontentsline{toc}{section}{ЗАКЛЮЧЕНИЕ}

В процессе выполнения данной работы была создана программная система, позволяющая сокращение размера исходного кода программ за счёт метапрограммирования.

Программная система предлагает к использованию возможности функциональной парадигмы и метапрограммирования для повышения скорости разработки, уменьшения объема кода, упрощения масштабирования и обеспечения большей мобильности разрабатываемого ПО.

Основные результаты работы:
\begin{enumerate}
	\item Проведён анализ предметной области.
	\item Спроектирован функциональный язык программирования с поддержкой метапрограммирования, являющийся подмножеством языка \quotes{Common Lisp}.
	\item Спроектирован интерпретатор этого языка.
	\item Выбраны технологии и методики для реализации интерпретатора.
	\item Интерпретатор реализован средствами языка программирования \quotes{C} и ОС \quotes{GNU/Linux}.
\end{enumerate}


Все требования, объявленные в техническом задании, были полностью реализованы, все задачи, поставленные в начале разработки проекта, были также решены.

Разработанный язык программирования реализует все основные концепции функциональной парадигмы и инструменты для метапрограммирования, представленные системой макросов. С помощью макросов разработаны различные синтаксические операторы: let, if, for, when, unless, case.

Разработанная программная система была успешно использована для сокращения исходного кода существующей программы путём её переписывания на разработанный язык с применением возможностей метапрограммирования.