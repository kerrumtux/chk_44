\addcontentsline{toc}{section}{СПИСОК ИСПОЛЬЗОВАННЫХ ИСТОЧНИКОВ}

\begin{thebibliography}{9}
	
	\bibitem{e1}	Пратт Т., Зелковиц М. Языки программирования: разработка и реализация / Под общей ред. А. Матросова. -- СПб.: Питер, 2002. -- 688 с.: ил. -- ISBN 5–318–00189–0. -- Текст~: непосредственный.
	\bibitem{e3}	Ахо, Альфред В., Лам, Моника С., Сети, Рави, Ульман, Джеффри Д. Компиляторы: принципы, технологии и инструментарий, 2 е изд . : Пер . с англ. -- М. : ООО “И.Д. Вильямс”, 2018 -- 1184 с. : ил. – ISBN 978–5–8459–1932–8 – Текст~: непосредственный.
	\bibitem{e4}	Свердлов С. З. Конструирование компиляторов. Учебное пособие // LAP Lambert Academic Publishing, 2015 -- 571 стр., ил. – ISBN 978–3–659–71665–2. -- Текст~: непосредственный.
	\bibitem{e5}	Хэзфилд Ричард, Кирби Лоуренс и др. Искусство программирования на С. Фундаментальные алгоритмы, структуры данных и примеры приложений. Энциклопедия программиста: Пер. с англ./Ричард Хэзфилд, Лоуренс Кирби и др. –К.: Издательство «ДиаСофт», 2001. -- 736 с. -- ISBN 966–7393–82–8. -- Текст~: непосредственный.
	\bibitem{e6}	Эмерик Ч., Карпер Б., Гранд К. Программирование на Clojure: Пер. с англ. Киселева А. Н. -- М.: ДМК Пресс, 2015. -- 816 с.: ил. -- ISBN 978-5-97060-299-7. -- Текст~: непосредственный.
	\bibitem{e7}	Керниган Б., Ритчи Д. Язык программирования Си. \textbackslash Пер. с англ., 3-е  изд., испр. -- СПб.: "Невский Диалект", 2001 -- 352 с: ил. -- ISBN5-7940-0045-7. -- Текст~: непосредственный.
	\bibitem{e8}	Клеменс, Бен. Язык С в XXI веке: Бен Клеменс; пер. с англ. А. А. Слинкина. -- Москва : ДМК Пресс, 2015. - 376 с. : ил. -- ISBN 978-5-97060-101-3. -- Текст~: непосредственный.
	\bibitem{e9}	Норманд Эрик. Грокаем функциональное мышление. -- СПб.: Питер, 2023. -- 608 с.: ил. -- ISBN 978-5-4461-1887-8. -- Текст~: непосредственный.    
	\bibitem{e10}	Сайбель П. Практическое использование Common Lisp / пер. с англ. А.Я. Отта. -- М.:ДМК Пресс, 2015. – 488 с.: ил. -- ISBN 978–5–94074–627–0. -- Текст~: непосредственный.    
	\bibitem{e11}	Фаулер, Мартин. Предметно-ориентированные языки программирования. : Пер. с англ. -- М. : ООО "И.Д. Вильямс", 2011. -- 576 с. : ил. -- ISBN 978-5-8459-1738-6. -- Текст~: непосредственный.    
	\bibitem{e12}	Структура и интерпретация компьютерных программ / Харольд Абельсон, Джеральд Джей Сассман, при участии Джули Сассман ; [пер. Г. К. Бронникова]. -- 2–е изд. – Москва : Добросвет : КДУ, 2012. – 608 с. : ил. -- ISBN 978–5–98227–829–6. – Текст~: непосредственный.    
	\bibitem{e13}	Лав Р. Linux. Системное программирование. 2–е изд. -- СПб.: Питер, 2014. -- 448 с.: ил. – ISBN 978–5–496–00747–4. -- Текст~: непосредственный.
	\bibitem{e14}	Хориков В. Принципы юнит-тестирования. -- СПб.: Питер, 2021. -- 320 с.: ил. -- ISBN 978-5-4461-1683-6. -- Текст~: непосредственный.
	\bibitem{e15} Пероцкая, В. Н. Основы тестирования программного обеспечения: учеб. пособие. -- Владимир: Изд-во ВлГУ, 2017. – 100 с. -- ISBN 978-5-9984-0777-2. -- Текст~: непосредственный.
	\bibitem{e16} Меджедович Д., Тахирович Э. Алгоритмы и структуры для массивных наборов данных / пер. с англ. А. В. Логунова. -- М.: ДМК Пресс, 2024. -- 340 с.: ил. -- ISBN 978-5-93700-250-1. -- Текст~: непосредственный.
	\bibitem{e17} Искусство программирования, том 1. Основные алгоритмы, 3-е изд. : Пер. с англ. -- М. : ООО "И.Д. Вильямс", 2018. - 720с. : ил. -- ISBN 978-5-8459-1984-7. -- Текст~: непосредственный.
	\bibitem{e18} Прата С. Язык программирования С++. Лекции и упражнения, 6-е изд. : Пер. с англ. -- М. : ООО "И.Д. Вильямc", 2012. -- 1248 с. : ил. -- ISBN 978-5-8459-1 778-2. -- Текст~: непосредственный.
	\bibitem{e19} Карпов В.Э. Теория компиляторов. Учебное пособие. 2-е изд., испр. и дополн. М., 2018. -- 92 с. -- ISBN 5–230–16344–5. -- Текст~: непосредственный.
	\bibitem{e20} Столяров А. В. Оформление программного кода: методическое пособие. -- М.: МАКС Пресс, 2012. -- 100 с. -- ISBN 978-5-317-04282-0. -- Текст~: непосредственный.
	\bibitem{e21} Бурмашева, Н. В. Лингвистические основы языка программирования С : учебное пособие / Н. В. Бурмашева ; Министерство науки и высшего образования Российской Федерации, Уральский федеральный университет. -- Екатеринбург : Изд-во Урал. ун-та, 2023. -- 86 с. : ил. -- 30 экз. -- ISBN 978-5-7996-3680-7. -- Текст~: непосредственный.
	\bibitem{e22} Грэм П. ANSI Common Lisp. -- Пер. с англ. -- СПб.: Символ-Плюс, 2012. -- 448 с., ил. -- ISBN 978-5-93286-206-3. -- Текст~: непосредственный.
	\bibitem{e23} Свердлов С. З. Языки программирования и методы трансляции: Учебное пособие. -- 2-е изд., испр. — СПб.: Издательство «Лань», 2019. — 564 с.: ил. -- ISBN 978-5-8114-3457-2. -- Текст~: непосредственный.
	\bibitem{e24} Bash. Карманный справочник системного администратора, 2-е и зд.: Пер. с англ. -- СпБ .: ООО “Альфа-книга”, 2017. — 152 с . : ил. -- ISBN 978-5-9909445-4-1. -- Текст~: непосредственный.
	\bibitem{e25} Командная строка Linux и автоматизация рутинных задач. -- СПб.: БХВ-Петербург, 2012. -- 352 с.: ил. -- ISBN 978-5-9775-0850-6. -- Текст~: непосредственный.
	\bibitem{e26} Буч Г., Рамбо Д., Якобсон И. Язык UML. Руководство пользователя. 2-е изд.: Пер. с англ. Мухин Н. – М.: ДМК Пресс. – 496 с.: ил. -- ISBN 5-94074-334-X. -- Текст~: непосредственный.
	\bibitem{e27} Языки и методы программирования: учебник для студентов учреждений высшего образования, обучающихся по направлениям подготовки "Прикладная математика и информатика", "Фундаментальная информатика и информационные технологии" / И. Г. Головин, И. А. Волкова. - 3-е изд., стер. -- Москва : Академия, 2018. - 303 с. : ил. -- ISBN 978-5-4468-6833-9. -- Текст~: непосредственный.
	\bibitem{e28} Гунько А.В. Системное программирование в среде Linux: учебное пособие / А.В. Гунько. -- Новосибирск: Изд-во НГТУ, 2020 – 235 с. -- ISBN 978-5-7782-4160-2. -- Текст~: непосредственный.
	\bibitem{e30} Функциональное программирование на F\#: учебное пособие для студентов технических вузов / Сошников Д. В. -- Москва : ДМК Пресс, 2011. -- 189 с. : ил. -- ISBN 978-5-94074-689-8. -- Текст~: непосредственный.
	\bibitem{e31} Сенлорен С., Эйзенберг Д. Введение в Elixir: введение в функциональное программирование / пер. с анг. А. Н. Киселева -- М.: ДМК Пресс, 2017. -- 262 с.: ил. -- ISBN 978-5-97060-518-9. -- Текст~: непосредственный.
	\bibitem{e32} Леоненков А. В. Самоучитель UML 2. -- СПб.: БХВ-Петербург, 2007. -- 576 с.: ил. -- ISBN 978-5-94157-878-8. -- Текст~: непосредственный.
\end{thebibliography}
