\addcontentsline{toc}{section}{СПИСОК ИСПОЛЬЗОВАННЫХ ИСТОЧНИКОВ}

\begin{thebibliography}{9}
	
	\bibitem{javascript} Пратт Т., Зелковиц М. Языки программирования: разработка и реализация / Под общей ред. А. Матросова. – СПб.: Питер, 2002. – 688 с.: ил. – ISBN 5–318–00189–0. – Текст~: непосредственный.
	\bibitem{php} Клинтон Л. Джеффери. Создайте свой собственный язык программирования. Руководство программиста по разработке компиляторов, интерпретаторов и доменно–ориентированных языков для решения современных вычислительных задач / пер. с англ. С. В. Минца. – М.: ДМК Пресс, 2023. – 408 с.: ил. – ISBN 978–5–93700–140–5. – Текст~: непосредственный.
	\bibitem{css} Ахо, Альфред В., Лам, Моника С., Сети, Рави, Ульман, Джеффри Д. Компиляторы: принципы, технологии и инструментарий, 2 е изд . : Пер . с англ. – М. : ООО “И.Д. Вильямс”, 2018 – 1184 с. : ил. – ISBN 978–5–8459–1932–8 – Текст~: непосредственный.
	\bibitem{mysql}	Свердлов С. З. Конструирование компиляторов. Учебное пособие // LAP Lambert Academic Publishing, 2015 – 571 стр., ил. – ISBN 978–3–659–71665–2. – Текст~: непосредственный.
	\bibitem{html5}	Хэзфилд Ричард, Кирби Лоуренс и др. Искусство программирования на С. Фундаментальные алгоритмы, структуры данных и примерыприложений. Энциклопедия программиста: Пер. с англ./Ричард Хэзфилд, Лоуренс Кирби и др. –К.: Издательство «ДиаСофт», 2001. – 736 с. – ISBN 966–7393–82–8. – Текст~: непосредственный.
	\bibitem{htmlcss} Костельцев А. В. Построение интерпретаторов и компиляторов : Использование программ BIZON, BYACC, ZUBR : [Учеб. пособие] / А. В. Костельцев. – СПб. : Наука и техника, 2001. – 218,[1] с. : ил. – ISBN 5–94387–033–4. – Текст~: непосредственный.
	\bibitem{bigbook} Шостак, Е. В. Основы программирования трансляторов языков программирования : учеб. – метод. пособие / Е. В. Шостак, И. М. Марина, Д. Е. Оношко. – Минск : БГУИР, 2019. – 66 с. : ил. – ISBN 978–985–543–470–3. – Текст~: непосредственный.
	\bibitem{uchiru} Коробова, И.Л. Основы разработки трансляторов в САПР : учебное пособие / И.Л. Коробова, И.А. Дьяков, Ю.В. Литовка. – Тамбов : Изд–во Тамб. гос. техн. ун–та, 2007 – 80 с. – ISBN 978–5–8265–0591–5. – Текст~: непосредственный.
	\bibitem{chaynik} Ричард Бёрд. Жемчужины проектирования алгоритмов: функциональный подход / Пер. с англ. В. Н. Брагилевского и А. М . Пеленицына. – М .; Д М К Пресс, 2013. – 330 с.: ил. – ISBN 978–5–94074–867–0. – Текст~: непосредственный.    
	\bibitem{22} Сайбель П. Практическое использование Common Lisp / пер. с англ. А.Я. Отта. – М.:ДМК Пресс, 2015. – 488 с.: ил. – ISBN 978–5–94074–627–0. – Текст~: непосредственный.    
	\bibitem{1231}	Форд Н. Продуктивный программист. Как сделать сложное простым, а невозможное – возможным. – Пер. с англ. – СПб.: Символ–Плюс, 2009. – 256 с., ил. – ISBN 978–5–93286–156–1. – Текст~: непосредственный.    
	\bibitem{sdf} Структура и интерпретация компьютерных программ [Текст] / Харольд Абельсон, Джеральд Джей Сассман, при участии Джули Сассман ; [пер. Г. К. Бронникова]. – 2–е изд. – Москва : Добросвет : КДУ, 2012. – 608 с. : ил. – ISBN 978–5–98227–829–6. – Текст~: непосредственный.    
	\bibitem{servsssds}	Лав Р. Linux. Системное программирование. 2–е изд. – СПб.: Питер, 2014. – 448 с.: ил. – ISBN 978–5–496–00747–4. – Текст~: непосредственный.
\end{thebibliography}
