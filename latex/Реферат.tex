\abstract{РЕФЕРАТ}

Объем работы равен \formbytotal{lastpage}{страниц}{е}{ам}{ам}. Работа содержит \formbytotal{figurecnt}{иллюстраци}{ю}{и}{й}, \formbytotal{tablecnt}{таблиц}{у}{ы}{}, \arabic{bibcount} библиографических источников и \formbytotal{числоПлакатов}{лист}{}{а}{ов} графического материала. Количество приложений – 2. Графический материал представлен в приложении А. Фрагменты исходного кода представлены в приложении Б.

Перечень ключевых слов: интерпретатор, функциональное программирование, метапрограммирование, Lisp, Common Lisp, подмножество Common Lisp, C, системное программирование, информационные технологии, сокращение исходного кода.

Объектом разработки является интерпретатор функционального языка программирования с поддержкой метапрограммирования.

Целью выпускной квалификационной работы является разработка программной системы, позволяющей сокращение размера исходного кода программ за счёт метапрограммирования.

В процессе создания интерпретатора были выделены основные сущности путем создания информационных блоков, использованы поля и методы модулей, обеспечивающие работу с сущностями предметной области, а также корректную работу интерпретатора. Был спроектирован функциональный язык программирования, являющийся подмножеством языка \quotes{Common Lisp}. С помощью макросов, реализующих метапрограммирование в этом языке, разработаны различные синтаксические операторы: let, if, for, when, unless, case. Программная система разрабатывалась для интерпретации спроектированного языка.

Разработанный интерпретатор был успешно использован для сокращения исходного кода существующей программы путём переписывания на разработанный язык с применением возможностей метапрограммирования.

\selectlanguage{english}
\abstract{ABSTRACT}
  
The volume of work is \formbytotal{lastpage}{page}{}{s}{s}. The work contains \formbytotal{figurecnt}{illustration}{}{s}{s}, \formbytotal{tablecnt}{table}{}{s}{s}, \arabic{bibcount} bibliographic sources and \formbytotal{числоПлакатов}{sheet}{}{s}{s} of graphic material. The number of applications is 2. The graphic material is presented in annex A. The layout of the site, including the connection of components, is presented in annex B.

List of keywords: interpreter, functional programming, metaprogramming, Lisp, Common Lisp, subset of Common Lisp, C, system programming, information technology, source code reduction.


The object of development is a functional programming language interpreter with metaprogramming support.

The purpose of the final qualification work is to develop a program system that allows reducing the size of the source code of programs due to metaprogramming.

In the process of creating the interpreter, the main entities were identified by creating information blocks, the fields and methods of the modules were used to ensure the work with the entities of the subject area, as well as the correct operation of the interpreter. A functional programming language, which is a subset of the \quotes{Common Lisp} language, was designed. With the help of macros realizing metaprogramming in this language, various syntactic operators were developed: let, if, for, when, unless, case. The program system was developed to interpret the designed language.

The developed interpreter was successfully used to reduce the source code of an existing program by rewriting it into the developed language with the use of metaprogramming capabilities.
\selectlanguage{russian}
