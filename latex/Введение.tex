\section*{ВВЕДЕНИЕ}
\addcontentsline{toc}{section}{ВВЕДЕНИЕ}

Парадигма метапрограммирования уже многие годы активно используется при разработке гибкого и адаптивного программного обеспечения, обеспечивая одни программы возможностью генерировать и трансформировать другие или самих себя, тем самым позволяя сокращать исходный код и одновременно с тем повышая его конфигурируемость и масштабируемость.

Со временем сложность программ возрастает, их функциональные возможности становятся более обширными, а требования к мобильности растут. Метапрограммирование может выступать как эффективный инструмент по снижению влияния таких требований на скорость и сложность разработки. В результате программисты могут писать ПО, формирующее программный код с учётом внешних условий и на основе предъявляемых требований.

Ещё одним популярным подходом, эффективным для решения всё тех же проблем и хорошо сочетаемым с генерацией кода, выступает функциональная парадигма программирования. Языки, берущие её за основу, приобрели значительную востребованность в последние годы благодаря своей способности обеспечивать более предсказуемый и прозрачный, легко анализируемый, модульный код. Поддержка функций высшего порядка и следование принципу неизменяемости данных также выделяют их как разумный вариант для разработок с такими требованиями.

Включая метапрограммирование в функциональный язык, производится заметное повышение потенциала языка к достижению поставленных разработкой целей \cite{e31}. Удачность симбиоза этих идей, наряду с другими успешными решениями, привела к тому, что языки, такие как Haskell, F\#, Erlang, OCaml, Lisp и Clojure стали превосходящими в индустрии, где требования к надежности и адаптивности систем высоки, а многие языки, изначально не имевшие таких возможностей, получают их частичную или полную поддержку \cite{e30}.

В этой выпускной квалификационной работе будет спроектирован функциональный язык программирования с поддержкой метапрограммирования и программная система для выполнения кода на этом языке - транслятор типа интерпретатор. Интерпретатор читает исходный код программы, анализирует и исполняет его инструкции, выводя результаты на экран или в файл. В отличие от компилятора \cite{e3}, который переводит весь код программы в машинный, интерпретатор обрабатывает его по одной инструкции, тем самым выполняя программы без предварительной компиляции. Язык, разработанный для этого интерпретатора, ставит минимализм и выразительность синтаксиса в приоритет, гарантируя, что основные концепции функционального программирования и метапрограммирования доступны и удобны в использовании.

Цель настоящей работы – разработка программной системы, позволяющей сокращение размера исходного кода программ за счёт метапрограммирования. Для достижения поставленной цели необходимо решить следующие задачи:
\begin{itemize}
	\item провести анализ предметной области;
	\item спроектировать функциональный язык программирования с поддержкой метапрограммирования;
	\item спроектировать интерпретатор этого языка;
	\item выбрать технологии и методики для реализации интерпретатора;
	\item реализовать интерпретатор средствами языка программирования \quotes{C} и ОС \quotes{GNU/Linux}.
\end{itemize}

\emph{Структура и объем работы.} Отчет состоит из введения, 4 разделов основной части, заключения, списка использованных источников, 2 приложений. Текст выпускной квалификационной работы равен \formbytotal{page}{страниц}{е}{ам}{ам}.

\emph{Во введении} сформулирована цель работы, поставлены задачи разработки, описана структура работы, приведено краткое содержание каждого из разделов.

\emph{В первом разделе} на стадии описания технической характеристики предметной области приводится сбор информации о технологиях и методиках, необходимых для реализации программной системы.

\emph{Во втором разделе} на стадии технического задания приводятся требования к разрабатываемому интерпретатору и языку.

\emph{В третьем разделе} на стадии технического проектирования представлены проектные решения для интерпретатора.

\emph{В четвертом разделе} приводится список модулей и их полей и методов, созданных при разработке, а также производится тестирование разработанного интерпретатора.

В заключении излагаются основные результаты работы, полученные в ходе разработки.

В приложении А представлен графический материал.
В приложении Б представлены фрагменты исходного кода. 
